\documentclass[12pt,oneside]{report}

% ============ PACKAGES ============
\usepackage[margin=1in]{geometry}
\usepackage{setspace}
\setstretch{1.15}

\usepackage{graphicx}
\usepackage{longtable,booktabs,array}
\usepackage{hyperref}
\usepackage{titlesec}
\usepackage{float}
\usepackage{caption}
\usepackage{ragged2e}

\hypersetup{
    colorlinks=true,
    linkcolor=black,
    urlcolor=blue
}

\newcolumntype{L}[1]{>{\RaggedRight\arraybackslash}p{#1}}

% ============ DOCUMENT ============
\begin{document}

% -------- COVER PAGE --------
\begin{titlepage}
\centering
\vspace*{3cm}

{\Huge \bfseries Software Design Document\\[0.6em]
We Go to Cal State LA}

\vspace{1.5cm}

{\Large Team 7 — Pacific Clinics}

\vspace{1.2cm}

{\Large December 2025}

\vfill
\end{titlepage}

\pagenumbering{roman}

% -------- TABLE OF CONTENTS --------
\tableofcontents
\clearpage

% -------- VERSION DESCRIPTION --------
\chapter*{Version Description}
\addcontentsline{toc}{chapter}{Version Description}

\begin{longtable}{L{2cm} L{9cm} L{3cm}}
\toprule
\textbf{Version} & \textbf{Description} & \textbf{Date} \\
\midrule
1.0 & Initial SDD created & Dec 2, 2025 \\
1.1 & Added System Architecture and UI sections & Dec 3, 2025 \\
1.2 & Completed Glossary and References & Dec 4, 2025 \\
2.0 & Final draft completed & Dec 10, 2025 \\
\bottomrule
\end{longtable}

\clearpage
\pagenumbering{arabic}

% ==============================
%       INTRODUCTION
% ==============================
\chapter{Introduction}

\section{Purpose}
This Software Design Document (SDD) describes the design, structure, and operation of the \textbf{We Go to Cal State LA} platform. It provides developers, reviewers, and stakeholders with a clear understanding of the system architecture, user interface, and database organization.

\section{Intended Audience}
This document is intended for:
\begin{itemize}
    \item Software developers building or maintaining the platform.
    \item Project managers and academic supervisors.
    \item Testers validating functionality.
    \item Stakeholders at Pacific Clinics.
\end{itemize}

\section{System Overview}
The platform provides:
\begin{itemize}
    \item Daily check-ins to support mental wellness.
    \item Access to categorized well-being resources.
    \item Surveys such as PHQ-9 and GAD-7.
    \item A chatbot for interactive assistance.
    \item Calendar events, reminders, and notifications.
\end{itemize}

The application is built using:
\begin{itemize}
    \item \textbf{React Native} for cross-platform mobile UI.
    \item \textbf{Cloudflare D1} for SQL-based data storage.
    \item \textbf{AWS Lambda / Cloudflare Workers} for serverless backend logic.
    \item \textbf{Amazon Cognito} for secure authentication.
\end{itemize}

% ==============================
%   SYSTEM ARCHITECTURE
% ==============================
\chapter{System Architecture}

\section{System Workflow}

The workflow of the system is shown below, capturing the interaction between users, external resources, video services, and internal logic.

\begin{figure}[H]
\centering
\includegraphics[width=0.95\textwidth]{yes.png}
\caption{External Entity Data Flow}
\end{figure}

\begin{figure}[H]
\centering
\includegraphics[width=0.95\textwidth]{teatea.jpg}
\caption{High-Level System Process Flow}
\end{figure}

\section{Component Breakdown}
The system is composed of the following major components:

\subsection*{Client-Side (Frontend)}
\begin{itemize}
    \item Implemented in React Native
    \item Handles UI rendering, user input, and navigation
    \item Communicates with backend through secure API requests
\end{itemize}

\subsection*{Server-Side (Backend)}
\begin{itemize}
    \item Runs on AWS Lambda and Cloudflare Workers
    \item Manages authentication, surveys, event scheduling, notifications
    \item Provides REST API endpoints
\end{itemize}

\subsection*{Database Layer}
\begin{itemize}
    \item Uses Cloudflare D1 (SQL relational model)
    \item Stores users, survey responses, resources, achievements, events
\end{itemize}

\begin{figure}[H]
\centering
\includegraphics[width=\textwidth]{snap3.png}
\caption{Entity–Relationship Diagram (ERD)}
\end{figure}

% ==============================
%       USER INTERFACE
% ==============================
\chapter{User Interface}

\section{How to Use the System}

\subsection*{Daily Check-In}
Users select an emotion and may receive a supportive message or recommended activity.

\subsection*{Resource Library}
Provides categorized well-being resources such as:
\begin{itemize}
    \item Mental health
    \item Financial well-being
    \item Academic support
\end{itemize}

\subsection*{Wellness Surveys}
Includes:
\begin{itemize}
    \item PHQ-9 (depression)
    \item GAD-7 (anxiety)
\end{itemize}

\subsection*{Chatbot}
AI-powered support for questions, guidance, and navigation.

\subsection*{Calendar / Events}
Users can:
\begin{itemize}
    \item View mental health events
    \item Add reminders
    \item Register for programs
\end{itemize}

\section{UI Screenshots}

\begin{figure}[H]
\centering
\includegraphics[width=0.95\textwidth]{snap1.PNG}
\caption{Example UI Screens: Check-In, Resources, Chatbot}
\end{figure}

\begin{figure}[H]
\centering
\includegraphics[width=0.95\textwidth]{snap2.PNG}
\caption{Example UI Screens: Inspiration, Surveys, Login}
\end{figure}

\section{User Interface Flow Model}

\begin{figure}[H]
\centering
\includegraphics[width=0.95\textwidth]{Flowchart.png}
\caption{System UI Flow Model}
\end{figure}

% ==============================
%       GLOSSARY
% ==============================
\chapter{Glossary}

\begin{longtable}{L{3cm} L{10cm}}
\toprule
\textbf{Acronym} & \textbf{Definition} \\
\midrule
UI & User Interface \\
API & Application Programming Interface \\
DB & Database \\
JWT & JSON Web Token \\
PWA & Progressive Web App \\
WCAG & Web Content Accessibility Guidelines \\
SQL & Structured Query Language \\
\bottomrule
\end{longtable}

% ==============================
%       REFERENCES
% ==============================
\chapter{References}

\begin{itemize}
    \item Cloudflare D1 Documentation — \url{https://www.cloudflare.com}
    \item React Native Documentation — \url{https://reactnative.dev}
    \item HIPAA Guidelines — U.S. Department of Health \& Human Services
    \item FERPA Guidelines — U.S. Department of Education
    \item WCAG Accessibility Standards — \url{https://www.w3.org/WAI}
\end{itemize}

\end{document}
