\documentclass[12pt]{article}
\usepackage{setspace}
\usepackage{titlesec}
\usepackage{geometry}
\usepackage{hyperref}
\usepackage{enumitem}

\geometry{margin=1in}
\setstretch{1.2}

\titleformat{\section}{\Large\bfseries}{\thesection}{1em}{}
\titleformat{\subsection}{\large\bfseries}{\thesubsection}{1em}{}

\begin{document}

\begin{center}
{\Large \textbf{README / User Manual \\ We Go to Cal State LA Application}}\\[1em]
Version 1.0
\end{center}

\section{Jira Project Board}
All team task management and sprint planning were completed in Jira.

\noindent\textbf{Jira Link:}  
\url{https://calstatela-cs3338.atlassian.net/jira/software/projects/CFP/boards/169}

\section{Formal Objective Breakdown}

\subsection*{Snapshot 1 – Start Objective}
\begin{itemize}
    \item Establish core system architecture.
    \item Set up frontend, backend, and database structure.
    \item Implement basic role-based authentication.
    \item Begin initial UI layout (Home, Categories, Activities).
    \item Create first drafts of SDD, SRS, and README.
\end{itemize}

\subsection*{Snapshot 2 – Checkpoint 1}
\begin{itemize}
    \item Added wellness surveys and response tracking.
    \item Implemented event calendar system.
    \item Expanded resource categories.
    \item Updated SDD, SRS, and README.
\end{itemize}

\subsection*{Snapshot 3 – Checkpoint 2}
\begin{itemize}
    \item Implemented Achievement Dashboard.
    \item Added AI-powered assistant.
    \item Added multilingual support (English \& Spanish).
    \item Updated all documentation.
\end{itemize}

\subsection*{Snapshot 4 – Final Checkpoint}
\begin{itemize}
    \item Improved UI/UX and accessibility.
    \item Completed final testing and fixes.
    \item Prepared final documentation and deliverables.
\end{itemize}

\section{Why This Software Matters}

The \textit{We Go to Cal State LA} application provides a centralized digital platform for students, families, and staff by combining academic tools, wellness assessments, resource libraries, achievement tracking, and event scheduling.

\begin{itemize}
    \item \textbf{Supports Academic Success:} Provides guided learning, study tools, and career development resources.
    \item \textbf{Promotes Wellness:} Includes emotional wellness surveys and reflective tools.
    \item \textbf{Engages Families:} Gives parents and guardians direct access to helpful information.
    \item \textbf{Ensures Accessibility:} Bilingual support and WCAG-compliant UI.
    \item \textbf{Centralizes Resources:} Brings multiple student services into one platform.
\end{itemize}

\section{How to Download or Access the Application}

\subsection{Requirements}
\begin{itemize}
    \item Node.js 18+ (for frontend and backend)
    \item npm or yarn
    \item PostgreSQL/MySQL or SQLite for development
\end{itemize}

\subsection{Running the Frontend}
\begin{verbatim}
cd frontend
npm install
npm run dev
\end{verbatim}
Access the app at: \url{http://localhost:3000}

\subsection{Running the Backend}
\begin{verbatim}
cd backend
npm install
npm run start
\end{verbatim}
API available at: \url{http://localhost:5000/api}

\subsection{Database Setup}

\subsubsection*{Create Database}
\begin{verbatim}
CREATE DATABASE wegotocsla;
\end{verbatim}

\subsubsection*{Migrate Schema}
\begin{verbatim}
npm run migrate
\end{verbatim}

\subsubsection*{Seed Demo Data}
\begin{verbatim}
npm run seed
\end{verbatim}

\subsection{Accessing the Application}
\begin{itemize}
    \item Student Portal: \url{http://localhost:3000}
    \item Parent Portal: \url{http://localhost:3000/parent}
    \item Staff Portal: \url{http://localhost:3000/staff}
\end{itemize}

\end{document}
